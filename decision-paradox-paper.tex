\documentclass[12pt, letterpaper]{article}

\usepackage[utf8]{inputenc}
\usepackage{amssymb}
\usepackage{hyperref}

\title{The Digital Era of Information Overflow and its Underlying "Choice Paradox"}
\author{Bruno Saboia de Albuquerque\thanks{Many thanks to my co-workers at the Bank for International Settlements (\url{https://bis.org}) for their immeasurably valuable comments and help.}\and Christophe Laforge}

\hypersetup{
	pdftitle={\@title},
	pdfauthor={\@author},
	pdfsubject={Essay about the paradox of choice},
	pdfcreator={TeXstudio},
	pdfkeywords={choice paradox},
	colorlinks=true,
	linkcolor=blue,
	citecolor=blue,
	filecolor=magenta,
	urlcolor=blue,
	bookmarksdepth=4
}

\parindent=0pt
\parskip=0.35cm

\date{March 2023}

\begin{document}
	\maketitle

	\begin{abstract}
		\noindent Established microeconomic theory asserts that a consumer with more choices is at least as well off as one with fewer—a concept reflected in the saying: "the more, the merrier." However, there are cases where more choices may actually harm consumers. Our goal is to investigate if—and when—having many choices jeopardizes consumer well-being and to devise optimal strategies given the overwhelming number of choices prevalent in the digital era of streaming platforms and information overflow.

		\noindent
		\\\textbf{Key-words}: Microeconomics. Consumer choice. Choice paradox.

	\end{abstract}
	\clearpage

	\section{Introduction}
	Intuitively, it seems better to have more choices rather than fewer. In developed market economies, consumers enjoy a multitude of options for goods and services, usually provided by competing firms. Consumers can presumably express their preferences more satisfactorily in such economies.

	This is seemingly preferable to the situation in underdeveloped economies, where limited markets restrict choices. No matter how wealthy an individual is, their quality of life cannot improve without meaningful choices.

	However, the internet era has overwhelmed consumers with a flood of options, creating an apparent paradox: in some cases, having \textit{too many} choices may be detrimental. Certain online services provide such an extensive range of choices that consumers struggle to make a selection. For example, on streaming platforms, the sheer volume of available films can make choosing what to watch a daunting task. Some users even abandon a service due to the overwhelming difficulty of selecting one option from hundreds of thousands—akin to searching for a needle in a haystack.

	Mathematical models of consumer choice in microeconomics have been instrumental in understanding consumer behavior. These models introduce concepts such as \textbf{preferences} and \textbf{utility}, which are valuable for researchers, educators, and analysts aiming to predict or explain consumer decisions.

	Unfortunately, these models seem inadequate in explaining why having many choices—such as in the case of streaming platforms—can be detrimental to a consumer. Many authors have explored this topic. For example, one study found that "people are more likely to purchase gourmet jams or chocolates or to undertake optional class essay assignments when offered a limited array of 6 choices rather than a more extensive array of 24 or 30 choices" \cite{IYENGAR_LEPPER_2000}. This suggests that there exists a limit beyond which additional options begin to harm decision-making.

	Our goal in this paper is to understand this phenomenon and model these circumstances within the same mathematical framework traditionally used in the microeconomic theory of consumer choice.

	\section{Preference Axioms}
	Preferences are typically described by a binary relation. The standard mathematical notation for expressing preference is $A \succsim B$\footnote{For our purposes, we do not distinguish between preference and strict preference relations.}.

	Consumer preferences are usually characterized by an axiomatic system. The two main axioms are \textit{completeness} and \textit{transitivity}:

	\begin{description}
		\item[Completeness]: Consumers can evaluate their preferences between different given alternatives. That is, given a finite number of choices, the consumer can rank them from best to worst.
		\item[Transitivity]: Consumer choice is consistent. If choice A is preferred to B and B is preferred to C, then A is also preferred to C.
	\end{description}

	Mathematically, let $X$ be the consumption set, where $x_i \in \mathbb{R}$ represents the number of units of commodity $i$\footnote{Commodity quantity is expressed in infinitely divisible units.}. Let $x^k= (x_1, x_2, \dots, x_n) \in X$ be the $k$-th consumption bundle, where $n$ is finite. Assume that $X \subseteq \mathbb{R}_+^n$, $0 \in X$, and that $X$ is closed and convex. Let $x^1, x^2, x^3 \in X$ be consumption bundles. Then, if:

	\begin{equation}
		x^1 \succsim x^2, \quad x^2 \succsim x^3
	\end{equation}

	it follows that:

	\begin{equation}
		x^1 \succsim x^3
	\end{equation}

	Although transitivity seems intuitive (e.g., if one prefers apples to oranges and oranges to bananas, one should prefer apples to bananas), empirical evidence suggests that real-world preferences are often intransitive \cite{GUADALUPELANAS2020e03459}.

	This raises questions about the completeness axiom. In reality, consumers may not always be able to determine the utility of an alternative choice. Many choices involve uncertainty, an aspect not captured by the traditional consumer choice model.

	\section{Utility and Choices}
	Utility functions typically represent preferences over bundles of multiple elements. However, in our case, let us simplify by assuming that each bundle consists of a single element.

	Let $u(x^k) \in \mathbb{R}_+^n$ be the utility function of the $k$-th choice, and let $U_k$ be the sum of utilities over a given choice set:

	\[ U_k = \sum_{n=1}^{k} u(x^n) \]

	We assume that \textit{utility is non-decreasing with additional choices}:

	\begin{equation}
		U_k \leq U_{k+j} \quad \forall k,j \in \mathbb{N}_{> 0}
	\end{equation}

	That is, a larger choice set should, in theory, yield at least as much utility as a smaller one. However, empirical findings challenge this assumption, suggesting that excessive choices may decrease utility due to cognitive overload.

	\section{Future Work: Incorporating Evaluation Penalties}
	The standard microeconomic model does not account for the cognitive costs of decision-making. Every additional choice requires time and effort to evaluate. When the number of options grows excessively, the time to assess them becomes prohibitively large, leading to what some call \textit{analysis paralysis}.

	We will now define a penalty function that quantifies this cognitive burden and investigate under what conditions excessive choice leads to net utility loss.

	\bibliographystyle{unsrt}
	\bibliography{decision-paradox-paper}
\end{document}
