\documentclass[12pt, letterpaper]{article}

\usepackage[utf8]{inputenc}
\usepackage{hyperref}
\hypersetup{
	%pagebackref=true,
	pdftitle={\@title},
	pdfauthor={\@author},
	pdfsubject={Essay about Bitcoins, with facts and questions},
	pdfcreator={LaTeX with abnTeX2},
	pdfkeywords={bitcoins criptocurrencies digital currencies},
	colorlinks=true,            % false: boxed links; true: colored links
	linkcolor=blue,             % color of internal links
	citecolor=blue,             % color of links to bibliography
	filecolor=magenta,          % color of file links
	urlcolor=blue,
	bookmarksdepth=4
}

\title{The digital era of information overflow and its underlying ``choice paradox''}
\author{Bruno Saboia de Albuquerque\thanks{Many thanks for my co-workers at the Bank for International Settlements (\url{https://bis.org}) for their immeasurably valuable help.}\and Christophe Laforge}

\date{December 2022}

\begin{document}
	\maketitle
	\begin{abstract}
		The classic microeconomics approach is to consider that more choices are always better for the customer -- a concept that has been conveyed before by folklore, with the well-known saying: ``the more, the merrier''. However, it might be the case that more choice is actually harmful for the consumer. Our goal is to provide a understanding of such cases.

		%\vspace{\onelineskip}
		\noindent
		\\\textbf{Key-words}: Microeconomics. Consumer choice.

		\begin{equation}
			\pi \times 2 = \frac{2}{2}
		\end{equation}
	\end{abstract}
\end{document}
